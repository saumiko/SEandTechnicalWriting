\documentclass[dvips,12pt]{article}

%\usepackage{hyperref}
\usepackage{graphicx}
\usepackage{textcomp}
\usepackage{fancyhdr}
\pagestyle{fancy}
\fancyhf{}


\setlength{\oddsidemargin}{0.25in}
\setlength{\textwidth}{6.5in}
\setlength{\topmargin}{0in}
\setlength{\textheight}{8.5in}


\begin{document}
\rhead{Chapter 24}
\lhead{A2: 2012331054}
\cfoot{\thepage}

\title{
  \textsc{\textbf{Quality Management}\\
  Chapter 24}
}
\author{
  Asif Mohaimen\\
  2012331054\\
}
\date{\today}

\maketitle

\section{Summary}
The main topics covered in this chapter are:
\begin{itemize}
\item Software quality.
\item Software standards.
\item Reviews \& inspections.
\item Software measurement \& metrics. 
\end{itemize}

The main emphasis was given on the minimal number of defects in the actual products (bug free if possible) to meet the standard of maintainability, reliability, portability etc. Software quality management defines the standard of process \& product and maintains the actual standard. Software quality assurance is the \emph{best practice} for maintaining the software standards. Quality managements should be documented in the organizational quality manual to meet the ISO 9001. \\

There is a team to check \& double check the quality standards. Their is a code inspection team to identify the bugs in codes and suggests the possible modes of corrections. Software measurements are used to gather data about software \& the processes. Product quality metrics are used to highlight the possible quality problems. The summarized architecture is shown in the figure.

\end{document}
