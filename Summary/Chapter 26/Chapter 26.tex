\documentclass[dvips,12pt]{article}


\usepackage[pdftex]{graphicx}
\usepackage{url}
%\usepackage{hyperref}
\usepackage{lipsum} % for filler text
\usepackage{fancyhdr}
\pagestyle{fancy}
\fancyhead{} % clear all header fields
\fancyhead[LE,RO]{Chapter 26 Summary}
\fancyhead[RE,LO]{A2: 2012331054}
\renewcommand{\headrulewidth}{0pt} % no line in header area
\fancyfoot{} % clear all footer fields
\fancyfoot[cf]{\thepage}           % page number in "outer" position of footer line
\fancyfoot[RE,LO]{}


\setlength{\oddsidemargin}{0.25in}
\setlength{\textwidth}{6.5in}
\setlength{\topmargin}{0in}
\setlength{\textheight}{8.5in}


\begin{document}


\title{
  \textsc{Process Improvement}\\
  Chapter 26
}
\author{
  \textsc{Asif Mohaimen}\\
  2012331054\\
}
\date{\today}

\maketitle


\section*{Summary}
\textsc{Software Process Improvement} (SPI) implies that, 
\begin{itemize}
\item elements of an effective software process can be defined in an effective manner.
\item an existing organizational approach to software development can be assessed against those elements, and
\item a meaningful strategy for improvement can be defined.
\end{itemize}
The SPI strategy transforms the existing approach to software development into something that is \emph{more focused, more repeatable, and more reliable} (in terms of the quality of the product produced and the timeliness of delivery).

\begin{figure}[b]
\begin{center}
\resizebox{3in}{!}{\includegraphics*{./img/ch261.jpg}}
\end{center} 
\caption{The process improvement cycle\label{ch261}}
\end{figure}

The Figure~\ref{ch261} shows the process improvement cycle which actually summarizes the whole concept of SPI. The principal ways of process improvement are agile approaches, geared to reducing process overheads, and maturity-based approaches based on better process management. Measurement using in defining the software process is also a vital part of SPI. 

A maturity model is applied within the context of an SPI framework. The CMMI process maturity model is an integrated process improvement model compatible with both staged \& continuous process development. The short summary of this process is given in this comic in Figure~\ref{comic}.
\begin{figure}[t]
\begin{center}
\resizebox{6in}{!}{\includegraphics*{./img/comic.jpg}}
\end{center} 
\caption{Intro to CMMI\label{comic}}
\end{figure}

Process improvement in the CMMI model is based on achieving a set of goals connected to \emph{good software engineering practice \& describing, standardizing and controlling} the practices used to achieve these goals. It includes recommended practices that might be used but this one is not mandatory.

\end{document}