\documentclass[dvips,12pt]{article}


\usepackage[pdftex]{graphicx}
\usepackage{url}
%\usepackage{hyperref}
\usepackage{lipsum} % for filler text
\usepackage{fancyhdr}
\pagestyle{fancy}
\fancyhead{} % clear all header fields
\fancyhead[LE,RO]{Chapter 17 Summary}
\fancyhead[RE,LO]{A2: 2012331054}
\renewcommand{\headrulewidth}{0pt} % no line in header area
\fancyfoot{} % clear all footer fields
\fancyfoot[cf]{\thepage}           % page number in "outer" position of footer line
\fancyfoot[RE,LO]{}


\setlength{\oddsidemargin}{0.25in}
\setlength{\textwidth}{6.5in}
\setlength{\topmargin}{0in}
\setlength{\textheight}{8.5in}


\begin{document}


\title{
  \textsc{Component-Based Software Engineering}\\
  Chapter 17
}
\author{
  \textsc{Asif Mohaimen}\\
  2012331054\\
}
\date{June 09, 2016}

\maketitle


\section*{Summary}
Component-based software engineering (CBSE), also known as component-based development (CBD), is a branch of software engineering that emphasizes the separation of concerns in respect of the wide-ranging functionality available throughout a given software system. The main topics covered in this chapter are:
\begin{itemize}
\item Components and component models 
\item CBSE processes
\item Component composition
\end{itemize}
There are some essential facts about component based software engineering. They are like - \emph{Independent components, Component standards,  Middleware \& a development process}.  It is a reuse based approach which literally defines \& implements loosely coupled components into systems. The model defines a standard to be followed by the providers \& the composers. It is a software component and it's procedures and functionality are defined within the interfaces of the component. There are two key things about it they are:
\begin{itemize}
\item{CBSE for reuse}
\item{CBSE with reuse}

\end{itemize}
\end{document}