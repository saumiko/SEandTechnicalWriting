\documentclass[dvips,12pt]{article}


\usepackage[pdftex]{graphicx}
\usepackage{url}
%\usepackage{hyperref}
\usepackage{lipsum} % for filler text
\usepackage{fancyhdr}
\pagestyle{fancy}
\fancyhead{} % clear all header fields
\fancyhead[LE,RO]{Chapter 21 Summary}
\fancyhead[RE,LO]{A2: 2012331054}
\renewcommand{\headrulewidth}{0pt} % no line in header area
\fancyfoot{} % clear all footer fields
\fancyfoot[LE,RO]{\thepage}           % page number in "outer" position of footer line
\fancyfoot[RE,LO]{}


\setlength{\oddsidemargin}{0.25in}
\setlength{\textwidth}{6.5in}
\setlength{\topmargin}{0in}
\setlength{\textheight}{8.5in}


\begin{document}


\title{
  \textsc{Service-oriented Architecture}\\
  Chapter 19
}
\author{
  \textsc{Asif Mohaimen}\\
  2012331054\\
}
\date{\today}

\maketitle


\section{Summary}
Service-oriented architectures (SOAs) are a way of developing distributed
systems where the system components are stand-alone services, executing on geographically distributed computers. The main topics discussed here are Services as reusable components, Service engineering \& Software development with services. SOA is an architecture where programs are constructed by encapsulating reusable functionality.  The principles of service-orientation are independent of any vendor, product or technology. SOA makes it easier for software components on computers connected over a network to cooperate. SOA is based on the concept of a service and these interfaces are defined in WSDL which includes a definition of the interface types and operations and these can be classified as utility services, business services or coordination services. This would normally include databases, software components, legacy systems, identity stores, XML schemas and any backing stores, e.g. shared directories. It is also beneficial to include any service agents employed by the service, as any change in these service agents would affect the message processing capabilities of the service. This includes composing and configuring services to create new composite services. Service oriented Architecture is a design concept that breaks down every application in modules ("service") that can be requested by a client application to fulfill a certain activity based on an input and optionally return a result. API stands for Application Programming Interface. Some enterprise architects believe that SOA can help businesses respond more quickly and more cost-effectively to changing market conditions.

\end{document}
