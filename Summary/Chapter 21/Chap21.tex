\documentclass[dvips,12pt]{article}


\usepackage[pdftex]{graphicx}
\usepackage{url}
%\usepackage{hyperref}
\usepackage{lipsum} % for filler text
\usepackage{fancyhdr}
\pagestyle{fancy}
\fancyhead{} % clear all header fields
\fancyhead[LE,RO]{Chapter 21 Summary}
\fancyhead[RE,LO]{A2: 2012331054}
\renewcommand{\headrulewidth}{0pt} % no line in header area
\fancyfoot{} % clear all footer fields
\fancyfoot[LE,RO]{\thepage}           % page number in "outer" position of footer line
\fancyfoot[RE,LO]{}


\setlength{\oddsidemargin}{0.25in}
\setlength{\textwidth}{6.5in}
\setlength{\topmargin}{0in}
\setlength{\textheight}{8.5in}


\begin{document}


\title{
  \textsc{Aspect-oriented Software Development}\\
  Chapter 21
}
\author{
  \textsc{Asif Mohaimen}\\
  2012331054\\
}
\date{\today}

\maketitle


\section{Summary}
Traditional software development focuses on decomposing systems into units of primary functionality, while recognizing that there are other issues of concern that do not fit well into the primary decomposition. \textsc{Aspect Oriented Software Development} (AOSD) is an approach of software development based on an aspect relatively known as a new type of abstraction. AOSD allows multiple concerns to be expressed separately and automatically unified into working systems. The first key thing discussed here is the separation of concerns which ensures the single task for each program element to ensure the understanding ability for that particular element. Thus, there were two types of concerns named- Stakeholder \& Cross-Cutting concerns to where there were Stakeholder concerns like- Functional, Quality of Service, Policy, System \& Organizational concerns in the topic of Stakeholder concerns and some aspects \& drawbacks of Cross cutting concerns into their respective sections. Then there were Aspects, join points and pointcuts. The focus of aspect-oriented software development is in the investigation and implementation of new structures for software modularity that provide support for explicit abstractions to modularize concerns. Respectively, A join point is a point in the execution of the program, which is used to define the dynamic structure of a crosscutting concern. The quantification over join points is expressed at the language level. This quantification may be implicit in the language structure or may be expressed using a query-like construct called a pointcut. In the software engineering integral part the key things were the discussion of the integration of this Aspect based development model in the software engineering process based on use cases in stakeholder concern. 

\end{document}
