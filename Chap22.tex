\documentclass[dvips,12pt]{article}


\usepackage[pdftex]{graphicx}
\usepackage{url}
%\usepackage{hyperref}
\usepackage{lipsum} % for filler text
\usepackage{fancyhdr}
\pagestyle{fancy}
\fancyhead{} % clear all header fields
\fancyhead[LE,RO]{Chapter 22 Summary}
\fancyhead[RE,LO]{A2: 2012331054}
\renewcommand{\headrulewidth}{0pt} % no line in header area
\fancyfoot{} % clear all footer fields
\fancyfoot[LE,RO]{\thepage}           % page number in "outer" position of footer line
\fancyfoot[RE,LO]{}


\setlength{\oddsidemargin}{0.25in}
\setlength{\textwidth}{6.5in}
\setlength{\topmargin}{0in}
\setlength{\textheight}{8.5in}


\begin{document}


\title{
  Project Management\\
  Chapter 22
}
\author{
  Asif Mohaimen\\
  2012331054\\
}
\date{\today}

\maketitle


\section{Introduction} \label{introduction}
Here in this specific chapter of the book the topic which is mainly discussed is the Management of Different issues that may rise during the project building phase.

The key things which are discussed in this chapter are listed below:
\begin{itemize}
\item Risk management
\item Managing people
\item Teamwork
\end{itemize}

\section{Risk Management Process}
This section basically deals with the process of identification, analysis, planning \& monitoring the risk factors that might rise during the project building phase. This portion of the chapter covers the basic details of different types of risks such as,
\begin{itemize}
\item Financial problems
\item Staff recruitment problems
\item Staff unavailability for natural phenomenons.
\item Version renewal
\item Major changes in specs.
\item Managerial changes.
\item Problems with database related issues etc.
\end{itemize}

\subsection{Risk Identification}
This portion deals with the process of the identification of different types of risks which are needed to be managed in the process of project management. The manager will have to categorize the list of risk factors. The manager can make a checklist of common risk factors in project building and take help from it to identify the risks in that particular project. The checklist can contain different categories like - \textit{Technological, People related, Organizational, Tools, Requirements, Estimation etc}. This will help the manager to resolve the issues caused by the risk factor.

\subsection{Risk Analysis}
This portion deals with the process of the categorization of the risk factors with different levels of priority. The manager calculates the probability of occurrence of each factors and classifies them as - \textit{very low, low, moderate, high or very high}. The manager should also label the consequences of the risks like - \textit{catastrophic, serious, tolerable or insignificant}.

\subsection{Risk Planning}
This portion describes the process of resolving the risk issues. The manager  has to consider each \& every risk \& has to think about strategies to manage that very risk. The strategies might be like they are shown below:
\begin{center}
\begin{tabular}{ |c|c| } 
 \hline
 \textbf{Types} & \textbf{Impacts} \\ 
 \textbf{Avoidance Strategy} & Risk arising probability will reduce. \\ 
 \textbf{Minimization Strategy} & Risk impact on project will reduce. \\
 \textbf{Contingency Plan} & Strategy to deal with the actual risk. \\  
 \hline
\end{tabular}
\end{center}

\subsection{Risk Monitoring}
The manager will have to monitor the risk factors and observe the changes of effects on the risk factors \& share them with the people in the group on the project meetings.

\section{Managing People}
This section basically deals with the management of the manpower working in the whole software building phase. The manager should essentially be people-oriented. A good people management is a key factor behind the complete success of a beautiful project.

\subsection{People Management Factors}
The manager should maintain a minimal level of etiquette with the team. He should ensure the following things in the people management works:
 \begin{itemize}
\item \textbf{Consistency:} There shouldn't be any discrimination between the project members of a project. He should abet all the employees to cope up with the process.
\item \textbf{Respect:} The project manager should respect the different knowledge levels of people working in the project. \emph{Admiration} is acceptable, \emph{abashing} isn't.
\item \textbf{Inclusion:} The project manager should ensure the participation of all of the team member in the building phase of the project. He is responsible for the abasement of the whole building process.
\item \textbf{Honesty:} The project manager should always be honest about any aberration or acclamation of the project building.
\end{itemize}

\subsection{Motivating People}
The manager should also motivate the manpower behind the project. He should try to manage to ensure the \emph{basic, personal \& social} needs of the people working in the project.

\subsection{Need Satisfaction}
In software development groups basic physiological and safety needs are not very much serious issues.They should be provided with \emph{Social, Esteem \& Self-realization} facilities inside the group.

\subsection{Personality Types} \label{ptype}
The manager should also maintain the consistency between different personality types inside the group. Where the types are basically of three types - \emph{Task, Self \& Interaction} oriented.

\subsection{Motivation Balance}
Project manager should balance his motivational skill for people according to class. It can change due to different phenomenons. Moreover, people are motivated because of being a part of the social group \& they work because they are moved by their partners.

\section{Teamwork}
This is the most important section for a group working on software development. A good software can't be developed by a person working alone. To develop a good project teamwork is needed very badly. Team spirit is a key thing for a group and this can be found in the motivations by people in the group. Group interaction is a key determinant of group performance. Flexibility is limited here.

\subsection{The Effectiveness of a Team}
As software development is a combination of diverse activities, different people with different ideas \& thinking can speed up the whole process of the development. The group needs to be well organized \& maintain technical and managerial communications to develop a good product.

\subsection{Selecting Group Members}
It is the responsibility of the manager to create a cohesive, organized group for effective outputs. It guarantees the right balance of technical skills and personalities for a effective working environment.

\subsection{Assembling a Team}
It might not be possible to form a good team because of the risk factors discussed earlier in section \ref{introduction}. Managers have to work within these constraints especially when there are shortages of trained staff.

\subsection{Group Composition}
Groups should be composed of different types of members as discussed in \ref{ptype}. 

\subsection{Group Organization}
The way that a group is organized affects the decisions that are made by that group, the ways that information is exchanged and the interactions between the development group and external project stakeholders. Small groups are usually organized informally without a rigid structure \& agile development is always based around an informal group . For large projects there might be hierarchical structures.

\subsection{Group Communications}
Communications within a group are influenced by factors such as the status of group members, the size of the group, the gender composition of the group, personalities and available communication channels. The manager has to ensure good communication between the members of the group.

\end{document}
